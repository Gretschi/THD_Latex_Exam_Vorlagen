\input{Leitungsvermittlung_commands.tex}
\question[5]

Nehmen Sie an, dass ein Netzwerk, welches mit Hilfe von Leitungsvermittlung
arbeitet, eine Datenrate von \datarate MBit/s aufweist. Darüber hinaus ist die
Leitung in \channels{} einzelne Kanäle aufgeteilt, so dass bis zu \channels{}
Teilnehmer gleichzeitig kommunizieren können.  Das Schalten der Leitung dauert
\switchingdelay ms. Wie lange dauert es für einen Teilnehmer um \data kBit
vollständig zu übertragen?

\textbf{Hinweise:} Die Ausbreitungsverzögerung soll in dieser Aufgabe vernachlässigt werden. Runden Sie Ihre Ergebnisse auf eine Stelle nach dem Komma. Stellen Sie den
Rechenweg nachvollziehbar dar um volle Punktzahl zu erhalten.

\begin{solutionbox}{5cm}
	\[
	\datarate MBit/s \equiv \datarate/\channels MBit/s = \channelrate MBit/s \quad\textrm{pro Kanal}
	\]

	Übertragungsverzögerung:
	\[ \frac{\data Bit}{\channelrate MBit/s} = \transdelay ms\]

	Gesamt: $\switchingdelay ms + \transdelay ms = \result ms$
\end{solutionbox}
