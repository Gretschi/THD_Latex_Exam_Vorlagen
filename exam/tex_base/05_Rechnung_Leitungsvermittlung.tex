\question[5]

Nehmen Sie an, dass ein Netzwerk, welches mit Hilfe von Leitungsvermittlung
arbeitet, eine Datenrate von \VAR{datarate} MBit/s aufweist. Darüber hinaus ist die
Leitung in \VAR{channels} einzelne Kanäle aufgeteilt, so dass bis zu \VAR{channels}
Teilnehmer gleichzeitig kommunizieren können.  Das Schalten der Leitung dauert
\VAR{switchingdelay} ms. Wie lange dauert es für einen Teilnehmer um \VAR{data} kBit
vollständig zu übertragen?

\textbf{Hinweise:} Die Ausbreitungsverzögerung soll in dieser Aufgabe vernachlässigt werden. Runden Sie Ihre Ergebnisse auf eine Stelle nach dem Komma. Stellen Sie den
Rechenweg nachvollziehbar dar um volle Punktzahl zu erhalten.

\begin{solutionbox}{5cm}
	\[
		\VAR{datarate} MBit/s \equiv \VAR{datarate}/\VAR{channels} MBit/s = \VAR{(datarate / channels)|round(2)} MBit/s \quad\textrm{pro Kanal}
	\]

	Übertragungsverzögerung:
	\[ \frac{\VAR{data} kBit}{\VAR{(datarate / channels)|round(2)} MBit/s} = \VAR{(data / (datarate / channels))|round(2)} ms\]

	Gesamt: $\VAR{switchingdelay} ms + \VAR{(data / (datarate / channels))|round(2)} ms = \VAR{(switchingdelay + data / (datarate / channels))|round(2)} ms$
\end{solutionbox}
